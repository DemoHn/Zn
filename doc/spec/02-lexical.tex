\documentclass{spec}

\chapter[Lexical Structure]{词法结构}

\paragraph{
  世上的编程语言,大抵采用纯文本的形式以存储代码。使用文本的优势不言自明:使用一般的文本编辑器即可阅读及编辑,
  同时亦方便托管到代码平台上进行维护,(LabView就是一个反例,其程序需要用专门的软件来查阅)\en{Zn}语言自然也不例外。
}

\paragraph{
  接下来的内容将会介绍\en{Zn}语言的词法结构。「词法结构」\enq{Lexical Structure}主要讲述的是如何用一些基本词素拼成一段\en{Zn}语言代码:
  所谓「连词成句」,大抵上就是这个意思。\en{Zn}语言主要由「标识符」\sectionrefq{Identifier},「关键字」\sectionrefq{Keyword},
  「立即数」\sectionrefq{Number},「文本」\sectionrefq{String},「注释」\sectionrefq{Comment},「换行及缩进」\sectionrefq{CRLF}等基本词素构成。
  这些词素的详细定义会在接下来的几个章节里给出。
}

\paragraph{
  中文编码曾经让无数程序员焦头烂额。有鉴于此,\en{Zn}语言统一要求\bold{所有程序文件须使用\enc{Unicode}字符集,以及使用\enc{UTF-8}编码}。
  \sectionrefq{Unicode}一节会详细介绍个中原因及具体细节。
}

\section[Unicode]{Unicode}

\paragraph{
  曾几何时,计算机上的中文编码并没有一个统一的标准:GB2312,GBK,GB18030,Big5等各自不同的编码标准使得「乱码」问题非常常见:
}

\subsection[UTF-8]{UTF-8}
Unicode 基本用法。毕竟在\en{Zn}语言中,只支持UTF-8编码。

\section[CRLF]{换行符}

\begin{BNF}
\nonterm{换行符} ::=
  \expr{\term{CR}}
  \expr{\term{LF}}
  \expr{\term{CR} \term{\LF}}
  \expr{\term{LF} \term{\CR}}
\end{BNF}

# 听说需要用EBNF定义换行符?(BNF的排版语法目测要自己设计一波?)

\section[WhiteSpaces]{空白符}

\begin{BNF}
\nonterm{空白符} ::= 
  \expr[or]{
    \term{\unicodechar{0009}}
    \term{\unicodechar{000B}}
    \term{\unicodechar{000C}}
    \term{\unicodechar{0020}}
    \term{\unicodechar{00A0}}
    \term{\unicodechar{2000}}
    \term{\unicodechar{2001}}
    \term{\unicodechar{2002}}
    \term{\unicodechar{2003}}
    \term{\unicodechar{2004}}
    \term{\unicodechar{2005}}
    \term{\unicodechar{2006}}
    \term{\unicodechar{2007}}
    \term{\unicodechar{2008}}
    \term{\unicodechar{2009}}
    \term{\unicodechar{200A}}
    \term{\unicodechar{200B}}
    \term{\unicodechar{202F}}
    \term{\unicodechar{205F}}
    \term{\unicodechar{3000}}
  }

大约显示成:
```
⟨空白符⟩ ::=
    U+0009 | U+000B | U+000C | U+0020 ...
```
\end{BNF}


\section[Indent]{缩进}

\section[Punctuation]{分隔符}

\section[Identifier]{标识符}

\section[Keyword]{关键字}

\section[QuoteIdentifier]{显式标识符}

\section[Number]{立即数}

\section[String]{文本}

\section[Comment]{注释}


