\documentclass{spec}

\chapter[Grammer]{语法小记}

\paragraph{
    我们真的需要用tex这么蛋疼的玩意来写文档嘛?是的!
    有个问题:option嵌套是否可以完成?
}

\p{
    原版LaTex语法设计得真的很学术界。。。太乱来了
    所以这里还是要支持嵌套的,否则这代码真的没法写
}
% assign variable
\begin{BNF}
    \label{定义变量及赋予初值}
    
<变量定义式> ::=
    | `令' <标识符列表> `为' <表达式>

<标识符列表> ::=
    | <标识符> <标识符后缀>

<标识符后缀> ::=    
    | `,' <标识符> <标识符后缀>
    | \epsilon
\end{BNF}

\p{可以试举几个例子:}
\code[block]{
令年号为“正德元年”
令年号@文本为“正德元年”
令甲,辛丑,正定为空
}

% 
\begin{BNF}
    \label{为变量赋值}

<赋值表达式> ::=
    | <标识符列表> `为' <表达式>

<标识符列表> ::=
    | <标识符> <标识符后缀>

<标识符后缀> ::=    
    | `,' <标识符> <标识符后缀>
    | \epsilon
\end{BNF}

\begin{BNF}
    \label{判断语句}
% TODO: 应该消除尾递归?
<判断表达式> ::=
    | <表达式> 为 <表达式>
    | <表达式> 是 <表达式>
\end{BNF}